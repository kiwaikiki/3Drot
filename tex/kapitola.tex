\chapter{Reprezentácie rotácií}

\label{kap:reprezentacie} % id kapitoly pre prikaz ref

V tejto kapitole si povieme niečo o možných reprezentáciach rotácií v 3D a o ich výhodách a prípadných limitáciach.

\subsection{Rotačné matice SO(3)} 
Za najzákladnejšiu reprezentáciu budeme považovať grupu rotačných 3x3 matíc $SO(3)$. 
Tieto matice sú ortogonálne a majú determinant alpha. Pre ortogonálne matice platí:
 QTQ = QQ-1 = QQT = In
 stĺpce tvoria ortonormálnu bázu priestoru teda stĺpce sú navzájom kolmé a majú veľkosť 1
 teda 
Okrem toho tvoria Lieho algebru(a bla bla bla, doplním ak to bude relevantné)
Navyše táto reprezentácia je spojitá nakoľko ňou dokážeme reprezentovať každú možnú rotáciu.
Táto reprezentácia je však nevhodná na použitie v neurónových sieťach, pretože neurónové siete pracujú s reálnymi číslami a nie s maticami. 
Ak by sme brali jednotlivé zložky matice ako premenné, mali by sme 9 parametrov, navyše by sa bolo treba uistiť o príslušnosti matice do SO(3).

Skladanie rotácií je dosiahnuté násobením dvoch matíc z SO3

\subsection{Eulerove uhly}
Na 3D rotáciu dá pozerať pozerať aj ako na 3 po sebe idúce rotácie okolo jednotlivých osí vlastných objektu. 
Teda je to násobok troch matíc, pričom každá má jeden parameter. Vo výsledku dostame niečo ako(tu ak by sme rotovali postupne podľa osí x, y a z):
<math>\Z_{1} Y_2 X_3 = \begin{bmatrix}
    cos_{\alpha} cos_{\beta} & cos_{\alpha} sin_{\beta} sin_{\gamma} - cos_{\gamma} sin_{\alpha} & sin_{\alpha} sin_{\gamma} + cos_{\alpha} cos_{\gamma} sin_{\beta} \\
    cos_{\beta} sin_{\alpha} & cos_{\alpha} cos_{\gamma} + sin_{\alpha} sin_{\beta} sin_{\gamma} & cos_{\gamma} sin_{\alpha} sin_{\beta} - cos_{\alpha} sin_{\gamma} \\
    - sin_{\beta} & cos_{\beta} sin_{\gamma} & cos_{\beta} cos_{\gamma} 
   \end{bmatrix}</math>
čo je rotačná matica z SO3, avšak má iba 3 parametre. 
Problémom tejto reprezentácie je, že neexistuje bijekcia medzi týmto priestorom a priestorom SO3, make
nakoľko pre niektoré matice z SO3 nastáva pri konverzií na uhly nejednoznačnosť.
Toto nám spoôsobuje nespojitosť (to bude vysvetlené až keď to pochopím...)
Napríklad v konkrétnej matici vyššie, ak cos_{\beta} je 0, tak alpha ani gamma nebudú jednoznačné. 
Táto situácia nastane ak uhol beta je 0. 
(tu neviem, ci je legit, ze beriem iba ten blizsi uhol, ak je mojim cielom mať to spojité...)


\subsection{Os-uhol}
Podľa Eulerovej rotačnej vety je každá rotácia v 3D priestore ekvivalentná rotácii okolo nejakej osi(vektoru) o určitý uhol.


\subsection{Quaternióny}

\subsection{Gramm schmidt}





V
zdrojovom kóde v súbore \verb'kapitola.tex' nájdete ukážky použitých
príkazov LaTeXu potrebných na písanie nadpisov a podnadpisov a
číslovaných a nečíslovaných zoznamov.

Text podkapitoly \ref{sec:jadro} je
prebratý zo smernice o záverečných prácach \cite[článok 5]{smernica} a popisuje typické členenie jadra práce (text medzi kapitolami Úvod a Záver). Hoci v niektorých študijných odboroch je vyžadované členenie práce na kapitoly uvedené v smernici, v informatike nie je nutné toto členenie dodržiavať a môžete text rozdeliť do kapitol podľa potrieb konkrétnej témy. Aj tak je však potrebné uviesť súčasný stav problematiky a z práce musí byť tiež jasný váš celkový prínos ako aj detaily vašej práce. Tu uvedené podkapitoly sú len na ukážku použitia príslušných príkazov v LaTeXu, vo vašej práci by ste mali spravidla nemali mať podkapitoly s textom iba na pár riadkov a mali by ste sa tiež vyvarovať prílišnému používaniu odrážkových alebo číslovaných zoznamov.

\section{Jadro práce podľa smernice}
\label{sec:jadro}
Jadro je hlavná časť školského diela a člení sa na kapitoly,
podkapitoly, odseky a pod., ktoré sa vzostupne číslujú.
Členenie jadra školského diela je určené typom  školského diela. Vo vedeckých 
a odborných prácach má jadro spravidla tieto hlavné časti:
\begin{itemize}
\item  súčasný stav riešenej problematiky doma a v zahraničí,
\item  cieľ práce,
\item  metodika práce a metódy skúmania,
\item  výsledky práce, 
\item  diskusia. 
\end{itemize}

\subsection{Súčasný stav}
V časti súčasný stav riešenej problematiky doma a v zahraničí autor uvádza 
dostupné informácie a poznatky týkajúce sa danej témy. Zdrojom pre spracovanie sú 
aktuálne publikované práce domácich a zahraničných autorov.  Podiel tejto časti práce 
má tvoriť približne 30 \% práce.

\subsection{Cieľ práce}
Časť cieľ práce  školského diela jasne, výstižne a presne charakterizuje predmet 
riešenia. Súčasťou sú aj rozpracované čiastkové ciele, ktoré podmieňujú dosiahnutie 
cieľa hlavného. 

\subsection{Metodika práce a metódy skúmania}
Časť metodika práce a metódy skúmania spravidla obsahuje:
\begin{enumerate}
\item  charakteristiku objektu skúmania,  
\item  pracovné postupy, 
\item  spôsob získavania údajov a ich zdroje, 
\item  použité metódy vyhodnotenia a interpretácie výsledkov,
\item  štatistické metódy.
\end{enumerate}

\subsection{Výsledky práce a diskusia}
Časti výsledky práce a diskusia sú najvýznamnejšími  časťami  školského diela. 
Výsledky (vlastné postoje alebo vlastné riešenia), ku ktorým autor dospel, sa musia 
logicky usporiadať a pri opisovaní sa musia dostatočne zhodnotiť. Zároveň sa 
komentujú všetky skutočnosti a poznatky v konfrontácii s výsledkami iných autorov. 
Výsledky práce a diskusia môžu tvoriť aj jednu samostatnú časť  a spoločne tvoria 
spravidla 30 až 40 \% školského diela.  

